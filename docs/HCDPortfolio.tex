%! suppress = MissingLabel

% Preamble
\documentclass[a4paper]{article}

\usepackage{geometry}
\geometry{
    a4paper,
    scale=0.8,
    bottom=2cm,
    top=2cm,
}

% Packages
\usepackage{amsmath}
\usepackage{csquotes}
\usepackage{natbib}
\usepackage{blindtext}
\usepackage{ctex}

\usepackage[utf8]{inputenc}
\usepackage[english]{babel}
\usepackage{indentfirst}
\usepackage{enumitem}
\usepackage{floatrow}
\usepackage{graphicx}
\usepackage{amsfonts}
\usepackage{amssymb}
\usepackage{listings}

% set up BNF generator
\usepackage{syntax}
\setlength{\grammarparsep}{10pt plus 1pt minus 1pt}
\setlength{\grammarindent}{10em}


\setlist{noitemsep} % removes spacing from items but leaves space around the whole list
%\setlist{nosep} % removes all vertical spacing within and around the list
\setlist[itemize]{topsep=0.25em, leftmargin=1.5pc}
\setlist[enumerate]{topsep=0.25em, leftmargin=1.5pc}

\renewcommand{\baselinestretch}{1.0}

\setlength{\parindent}{0em}
\graphicspath{ {./} }

%! suppress = EscapeUnderscore
\newcommand*{\tproject}[1]{
    \begin{math}
        \pi_{#1}
    \end{math}
}

%! suppress = EscapeUnderscore
\newcommand*{\tselect}[1]{
    \begin{math}
        \sigma_{#1}
    \end{math}
}

\newcommand{\emptyline}{\\ \hspace*{\fill} \\}

\newcommand*{\textbfit}[1]{\textbf{\textit{#1}}}

\newcommand*{\seqdef}[1][a]{$({#1}_n)_{n\geq1}$}
\newcommand*{\subseqdef}[1][a]{$({{#1}_n}_i)_{i\geq1}$}
\newcommand*{\seq}[1][a]{#1_n}
\newcommand*{\limtoinf}[1][n]{\lim_{#1\rightarrow\infty}}
\newcommand{\abs}[1]{\left| #1 \right|}

\newcommand{\shell}[1]{\lstinline!#1!}
\usepackage{sourcecodepro}
\usepackage[T1]{fontenc}

\usepackage{hyperref}
\usepackage{xcolor}
\usepackage{textcomp}
\hypersetup{
    colorlinks,
    linkcolor={blue!50!black},
    citecolor={blue!50!black},
    urlcolor={blue!80!black}
}

\lstset{breaklines=true,basicstyle=\ttfamily\small,autogobble=true,language=C}

% Document
%! suppress = TooLargeSection
%! suppress = Quote
\begin{document}
    \title{
        \vspace{-3em}
        Human Centred Design Techniques Portfolio}
    \author{
        Group 43
    }
    \date{\vspace{-2em}}
    \maketitle

    \subsection*{Impact for Target Audience}

    Interviews were conducted to understand how Solvo can affect the stakeholders.

    We came back to students who used to discuss only with friends.
    The users responded with the following quotes:
    \begin{itemize}
        \item \textit{``Do you think you would be willing to use it if you understood this feature (the thoughts)?''}
        \begin{itemize}
            \item[-] ``Yeah, if I only have some general ideas, I will definitely not post an answer.
            But if I know that I can post `Thoughts', I am glad to share them.''
        \end{itemize}

        \item \textit{``Do you think you will be getting more feedback with the reaction system?''}
        \begin{itemize}
            \item[-] ``Sure.''
            \item[-] ``For me, I will write a comment only when I think the answer is incorrect.
            If I agree with the answer, I won't say `You are correct', which takes time.''
        \end{itemize}

        \item
        \textit{``As a user, are you willing to give more feedback using the reaction system?''}
        \begin{itemize}
            \item[-] ``Yeah absolutely.''
            \item[-] ``I don’t want the comments to be filled up with meaningless comments like `+1'.
            With this reaction system, it can be much more clean.''
        \end{itemize}
        \item \ldots
    \end{itemize}

    As it can be summarised from the quotes:

    On Solvo, students who used to discuss only with friends can now share ideas with more people, and receive feedback from peers.
    Students can therefore gain feedback from the reactions and comments, knowing if they are appreciated by others,
    and learn what they can't learn in the past.

    \subsection*{Impact for Other Stakeholders}

    We interviewed five tutors about their thoughts on Solvo.

    \begin{itemize}
        \item \textit{``Will you welcome your institution to introduce Solvo to students?''}
        \begin{itemize}
            \item[-] ``Of course.''
            \item[-] ``I love the peer discussion it (Solvo) provides.
            As a tutor, we always encourage students to discuss with each other and answer questions on the forum, even if it might be wrong.
            This allows them to learn from each other.''
        \end{itemize}
    \end{itemize}

    It is known from the interviews that tutors like the way Solvo promotes peer discussion, and are happy to welcome the introduction of Solvo.

\end{document}
